% !TeX program = xelatex
% !TeX encoding = utf8
% !TeX root = RegT1E_HS22.tex

%% TODO: publish to CTAN
\documentclass[margin=normal]{tex/hsrzf}

%%%%%%%%%%%%%%%%%%%%%%%%%%%%%%%%%%%%%%%%%%%%%%%%%%%
% Packages
\usepackage{multicol}
\usepackage[export]{adjustbox}
\usepackage{bm}
\usepackage{color, colortbl}
\usepackage{trfsigns}
\usepackage{graphicx}
\usepackage{tabularx}
\usepackage{mathrsfs}
\usepackage{tikz}
\usepackage{xcolor}
\usepackage{hyperref}

\usetikzlibrary{plotmarks}
\usepackage{pdfpages}
\usepackage{pgfplots}
\usepackage{lscape}


\usetikzlibrary{plotmarks}

\definecolor{TabularBackgroundColor}{rgb}{0.83,0.96,0.96}
\definecolor{TabularTitleColor}{rgb}{0.89,0.94,0.94}
\definecolor{RefColor}{rgb}{0.2,0.47,0.39}

\newcommand{\skript}[1]{{\scriptsize \color{RefColor}Skript S.#1}}
\newcommand{\titlewithref}[2]{\texorpdfstring{#1 {\scriptsize \color{RefColor}Skript S.#2}}{#1}}
%% TODO: publish to CTAN
\usepackage{tex/hsrstud}

%% Language configuration
\usepackage{polyglossia}

\setdefaultlanguage[variant=swiss]{german}

%% License configuration
\usepackage[
    type={CC},
    modifier={by-nc-sa},
    version={4.0},
    lang={german},
]{doclicense}


%%%%%%%%%%%%%%%%%%%%%%%%%%%%%%%%%%%%%%%%%%%%%%%%%%%
% Metadata

\course{Elektrotechnik}
\module{RegT1E}
\semester{Herbstsemester 2022}

\authoremail{joel.leirer@ost.ch}
\author{\textsl{Joël Leirer} -- \texttt{\theauthoremail}}

% did someone help you with this work?
\contributors{
  % do not forget to add yourself!
}

\title{\texttt{\themodule} Zusammenfassung}
\date{\thesemester}

%%%%%%%%%%%%%%%%%%%%%%%%%%%%%%%%%%%%%%%%%%%%%%%%%%%
% Document

\begin{document}

% use roman numberals for introductiory pages
\pagenumbering{roman}

\maketitle


% show the names of the people who contributed to this document.
% \section*{Contributors}
% \thecontributors

\section*{Lizenz}
\doclicenseThis

\clearpage
\tableofcontents

% actual content
\clearpage
\setcounter{page}{1}
\pagenumbering{arabic}
\section{Allgemeines}
\subsection{Begriffe}
\small
\begin{itemize}
      \item Prozess
            \begin{itemize}
                  \item Die Gesamtheit zusammenwirkender Vorgänge, welche durch
                        die Materie, Energie und Information
                        umgefort, transportiert und gespeichert wird.
            \end{itemize}

      \item System
            \begin{itemize}
                  \item Ist gegenüber der Umwelt abgegrenzt, hat Eingänge,
                        Ausgänge (und Zustand).
                  \item (LTI-/)LZI-Systeme: Lineare-ZeitInvariante Systeme
                        (Linearität gilt, System ist unabhängig von zeitlicher Verschiebung
                        --> DGL mit Konst. Koeff.)
            \end{itemize}

      \item Modell
            \begin{itemize}
                  \item Beschreibung von Systemen, wird genutzt für
                        Erklärung, Prognose, Gestaltung und Optimierung.
                  \item Es gibt kein "richtiges Modell",
                        ein Modell beschreibt ein System nur so genau wie nötig.
            \end{itemize}
      \item Modellieren
            \begin{itemize}
                  \item Teilsysteme erstellen und diese
                        in weitere Teilsysteme aufzutrennen.
                        So erhält man einfache Grundsysteme (Grundglieder),
                        welche sich einfach Mathematisch beschreiben lassen.
                        Zusätzlich kehern einige Teilsysteme in ihrer
                        Struktur oft wieder und können wiederverwendet werden.

                  \item Top-Down: System in Teilsysteme teilen
                  \item Bottom-Up: System aus Grundglieder aufbauen
            \end{itemize}
      \item Grundglieder
            \begin{itemize}
                  \item Kleinste Teilsysteme. Sie können weiter augeteilt werden, siehe Abschnitt Grundglieder
            \end{itemize}
      \item Sprungantwort
            \begin{itemize}
                  \item Reaktion des Systems auf die Sprungfunktion. Siehe LTI-Systeme

            \end{itemize}
      \item Schrittantwort
      \item \begin{itemize}
                  \item Reaktion des Systems auf die Schrittfunktion. Siehe LTI-Systeme
            \end{itemize}
      \item Ausgleich
            \begin{itemize}
                  \item Prozesse ohne Ausgleich: Sprungantwort wächst Grenzenlos an
                  \item Prozess mit Ausgleich: Sprungantwort strebt endlichem Wert zu
            \end{itemize}
\end{itemize}
\normalsize

\subsection{\titlewithref{Regelkreis}{9}}
\includegraphics[height = 4cm]{img/Regelkreis.png}


\subsection*{\titlewithref{Steuerung}{18}}
\subsection*{\titlewithref{Regelung}{17}}
\subsection*{\titlewithref{Gegen- und Mitkopplung}{106}}


%%%%%%%%%%%%%%%%%%%%%%%%%%%%%%%%%%%%%%%%%%%
%Packages
%%%%%%%%%%%%%%%%%%%%%%%%%%%%%%%%%%%%%%%%%%%
%Content
%%%%%%%%%%%%%%%%%%%%%%%%%%%%%%%%%%%%%%%%%%%

\begin{landscape}
    \subsection{Grundglieder}
    \begingroup
    \scriptsize
    \newcommand{\ImageWidth}{70pt}
    \begin{tabularx}{\linewidth}{|p{100pt}|p{160pt}|p{60pt}|p{80pt}|p{120pt}|p{80pt}|}
          \hline
          \textbf{Benennung}
           &
          \textbf{Funktion}
           &
          \textbf{UTF\textsuperscript{1}}
           &
          \textbf{Symbol}
           &
          \textbf{Sprungantwort}
           &
          \textbf{Plot}
          \\
          \hline
          \hline
          %%%%%%%%%%%%%%%%%%%%%%%%%%%%%%%%%%%%%%%%%%%%%%%%%%%%%%%%%%%%%%%%%%%%%%%%%%%%%%%%%%%%%%%%%%%%%
          \textbf{P-Glied\textsuperscript{2}}
          \newline Proportionalglied
           &
          $y = K \cdot u$
           &
          $K$
           &
          \raisebox{-.5\height}{\includegraphics[width = \ImageWidth]{img/DIN-Symbole/Proportionalglied.png}}
           &
          $K$
           &
          \raisebox{-.5\height}{
                \resizebox{\ImageWidth}{!}{%
                      \begin{tikzpicture}
                            % Grid
                            \draw[help lines,dashed] (0,0) grid (5,3);

                            % Axes
                            \draw[very thick,latex-latex] (0,3.25) node[left]{$y(t)$}
                            |- (5.25,0) node[below]{$t$};

                            % Plot function
                            \draw[ultra thick,teal] (-0.5,0) node[left,black](s0){$0$}
                            -- ++(0.5,0)
                            plot[domain=0:5,
                                        samples = 50,
                                        smooth]({\x}, {2});
                      \end{tikzpicture}
                }
          }
          \\
          \hline
          \rowcolor{TabularBackgroundColor}
          %%%%%%%%%%%%%%%%%%%%%%%%%%%%%%%%%%%%%%%%%%%%%%%%%%%%%%%%%%%%%%%%%%%%%%%%%%%%%%%%%%%%%%%%%%%%%
          \textbf{I-Glied}
          \newline(Idealer Integrierer)
           &
          $y(t) = K \cdot \int \limits _{t=0} ^{t} u(\tau) d\tau + y(0)$
          \newline $\dot{y}(t) = K \cdot u(t)$
           &
          $K \frac{1}{s}$
           &
          \raisebox{-.5\height}{\includegraphics[width = \ImageWidth]{img/DIN-Symbole/Integrator.png}}

           &
          $K \cdot t$
           &
          \raisebox{-.5\height}{
                \resizebox{\ImageWidth}{!}{%
                      \begin{tikzpicture}
                            % Grid
                            \draw[help lines,dashed] (0,0) grid (5,3);

                            % Axes
                            \draw[very thick,latex-latex] (0,3.25) node[left]{$y(t)$}
                            |- (5.25,0) node[below]{$t$};

                            % Plot function
                            \draw[ultra thick,teal] (-0.5,0) node[left,black](s0){$0$}
                            -- ++(0.5,0)
                            plot[domain=0:5,
                                        samples = 50,
                                        smooth]({\x},);
                      \end{tikzpicture}
                }
          }
          \\
          \hline
          %%%%%%%%%%%%%%%%%%%%%%%%%%%%%%%%%%%%%%%%%%%%%%%%%%%%%%%%%%%%%%%%%%%%%%%%%%%%%%%%%%%%%%%%%%%%%
          \textbf{Totzeit-Glied}
           &
          $y(t) = u(t-T_t)$
           &
          $e^{-s T_t}$
           &
          \raisebox{-.5\height}{\includegraphics[width = \ImageWidth]{img/DIN-Symbole/Totzeitglied.png}}
           &
          $\sigma(t-T) = \begin{cases}
                                     0, t<T     \\
                                     1, t\geq T \\
                               \end{cases}$
           &
          \raisebox{-.5\height}{
                \resizebox{\ImageWidth}{!}{%
                      \begin{tikzpicture}
                            % Grid
                            \draw[help lines,dashed] (0,0) grid (5,3);

                            % Axes
                            \draw[very thick,latex-latex] (0,3.25) node[left]{$y(t)$}
                            |- (5.25,0) node[below]{$t$};

                            % Plot function
                            \draw[ultra thick,teal] (-0.5,0) node[left,black](s0){$0$}
                            -- ++(0.5,0)
                            plot[domain=0:5,
                                        samples = 100,
                                  ]({\x},{ (\x<=2) * 0  + (\x>2) * 2});
                      \end{tikzpicture}
                }
          }
          \\
          \hline
          \rowcolor{TabularBackgroundColor}
%%%%%%%%%%%%%%%%%%%%%%%%%%%%%%%%%%%%%%%%%%%%%%%%%%%%%%%%%%%%%%%%%%%%%%%%%%%%%%%%%%%%%%%%%%%%%
          \textbf{D-Glied}
          \newline Idealer Differenzierer
           &
          $y(t) = K \cdot \frac{du(t)}{dt} = K\cdot \dot{u}(t)$
           &
          $K \cdot S $
           &
          \raisebox{-.5\height}{\includegraphics[width = \ImageWidth]{img/DIN-Symbole/D-Glied.png}}
           &
          $K\cdot \delta(t)$
           &
          \raisebox{-.5\height}{
                \resizebox{\ImageWidth}{!}{%
                      \begin{tikzpicture}
                            % Grid
                            \draw[help lines,dashed] (0,0) grid (5,3);

                            % Axes
                            \draw[very thick,latex-latex] (0,3.25) node[left]{$y(t)$}
                            |- (5.25,0) node[below]{$t$};

                            % Plot function
                            \draw[ultra thick ,teal] (2,0) -- node[left]{$\delta$} (2,2);
                            
                      \end{tikzpicture}
                }
          }
          \\
          \hline
          %%%%%%%%%%%%%%%%%%%%%%%%%%%%%%%%%%%%%%%%%%%%%%%%%%%%%%%%%%%%%%%%%%%%%%%%%%%%%%%%%%%%%%%%%%%%%
          \textbf{DT\textsubscript{1}-Glied}
          {
                \tiny 
                \newline Realisierbarer Differenzierer
                \newline K: Verstärkung
                \newline T: Zeitkonstante
          }
           &
          $ y(t) + T \cdot \dot{y}(t) = K \dot{u}(t)$
          \newline $y(t) = K \dot{u}(t) - T \cdot \dot{y}(t)$
          
           &
          $\frac{K \cdot s}{1+T_s}$
           &
          \raisebox{-.5\height}{\includegraphics[width = \ImageWidth]{img/DIN-Symbole/DT-Glied.png}}
           &
          $\frac{K}{T}\cdot e^{-\frac{t}{T}}$
           &
          \raisebox{-.5\height}{
                \resizebox{\ImageWidth}{!}{%
                      \begin{tikzpicture}
                            % Grid
                            \draw[help lines,dashed] (0,0) grid (5,3);

                            % Axes
                            \draw[very thick,latex-latex] (0,3.25) node[left]{$y(t)$}
                            |- (5.25,0) node[below]{$t$};

                            % Plot function
                            \draw[ultra thick,teal] (-0.5,0) node[left,black](s0){$0$}
                            -- ++(0.5,0)
                            plot[domain=0:5,
                                        samples = 50,
                                        smooth]({\x},{2.5*exp(-(\x))});
                      \end{tikzpicture}
                }
          }
          \\
          \hline   
          \rowcolor{TabularBackgroundColor}
          %%%%%%%%%%%%%%%%%%%%%%%%%%%%%%%%%%%%%%%%%%%%%%%%%%%%%%%%%%%%%%%%%%%%%%%%%%%%%%%%%%%%%%%%%%%%%
          \textbf{PT\textsubscript{1}-Glied}
          {
                \tiny \newline K: Verstärkung
                \newline T: Zeitkonstante
          }
           &
          $ T\dot{y} + y = K u(t)$
          \newline $T= \frac{1}{K_I \cdot K_P} $
          \newline $K = \frac{1}{K_P}$
           &
          $\frac{K}{1+T_s}$
           &
          \raisebox{-.5\height}{\includegraphics[width = \ImageWidth]{img/DIN-Symbole/PT1-Glied.png}}
           &
          $K (1-e^{-\frac{t}{T}})$
           &
          \raisebox{-.5\height}{
                \resizebox{\ImageWidth}{!}{%
                      \begin{tikzpicture}
                            % Grid
                            \draw[help lines,dashed] (0,0) grid (5,3);

                            % Axes
                            \draw[very thick,latex-latex] (0,3.25) node[left]{$y(t)$}
                            |- (5.25,0) node[below]{$t$};

                            % Plot function
                            \draw[ultra thick,teal] (-0.5,0) node[left,black](s0){$0$}
                            -- ++(0.5,0)
                            plot[domain=0:5,
                                        samples = 50,
                                        smooth]({\x},{2.5*(1- exp(-(\x)))});
                      \end{tikzpicture}
                }
          }
          \\
          \hline
          %%%%%%%%%%%%%%%%%%%%%%%%%%%%%%%%%%%%%%%%%%%%%%%%%%%%%%%%%%%%%%%%%%%%%%%%%%%%%%%%%%%%%%%%%%%%%
          \textbf{PT\textsubscript{2}-Glied}
          {\tiny
                \newline K: Verstärkung
                \newline T: Zeitkonstante
                \newline $\zeta$: Dämpfungskonstante
          }
           &
          $T^2 \cdot \ddot{y}(t) + 2 \zeta T \cdot \dot{y}(t) + y(t) = K \cdot u(t)$
          \newline $y(t) = K \cdot x(t) + T^2 \cdot (-\ddot{y}(t)) - 2 \zeta T \cdot \dot{y}(t)$
          {\tiny
                      \newline $\zeta = \frac{h}{\sqrt{h^2 + \pi^2}}$
                      \newline $h = \log(\frac{y_m}{y_{\infty}})$
                }
           &
          $\frac{K}{T^2 + s^2 + 2 \zeta T s + 1}$
           &
          \raisebox{-.5\height}{\includegraphics[width = \ImageWidth]{img/DIN-Symbole/PT2-Glied.png}}
           &
          $KA(1+e^{\sigma t}(-cos(\omega t) + \frac{\sigma}{\omega}sin(\omega t)))$
          {\tiny
                      \newline $\omega = \frac{2\pi}{T_m}$
                      \newline $K = \frac{y_{\infty}}{A} $
                      \newline $\sigma = \frac{h\omega}{\pi}$
                }
           &
          \raisebox{-.5\height}{
                \resizebox{\ImageWidth}{!}{%
                      \begin{tikzpicture}
                            % Grid
                            \draw[help lines,dashed] (0,0) grid (5,3);

                            % Axes
                            \draw[very thick,latex-latex] (0,3.25) node[left]{$y(t)$}
                            |- (5.25,0) node[below]{$t$};

                            % Plot function
                            \draw[ultra thick,teal] (-0.5,0) node[left,black](s0){$0$}
                            -- ++(0.5,0)
                            plot[domain=0:5,
                                        samples = 50,
                                        %TODO: Fix this!
                                        smooth]({\x},{2.5  *   (1- exp(-3*(\x))  *  (- cos(deg(1.9848*\x)) + (-3/1.9848)*sin(deg(1.9848*\x)))  });
                            %KA = 2.5, Sigma = -3, Omega = 1.9848
                      \end{tikzpicture}
                }
          }
          \\
          \hline       
    \end{tabularx}
    \tiny{
          \\
          1. UTF = Übertragungsfunktion (Laplacetransformierte Sprungantwort)\\
          2. Proportionalglied ist einziges Statisches glied.\\
    }
    \endgroup
    \normalsize
\end{landscape}


\section{\titlewithref{Klassifizierung von Systemen}{76,78,81}}
\begin{tabular}{|p{0.5\textwidth}|p{0.5\textwidth}|}
      \hline
      \rowcolor{TabularTitleColor}
      \textbf{Statisch}      & \textbf{Dynamisch}    \\
      \begin{itemize}
            \item System ohne Gedächtnis
            \item Ausgang hängt nur von aktuellem Eingang ab.
            \item $y(t) = f(x(t)) \forall t$
      \end{itemize}
                             &
      \begin{itemize}
            \item System mit Gedächtnis
            \item Ausgang hängt von aktuellem sowie vergangenen und zukünftigen Eingängen ab.
            \item $y(t) = f(x(t \pm t_0)) $
      \end{itemize}
      \\
      \hline
      \rowcolor{TabularTitleColor}
      \textbf{Kausal}        & \textbf{Akausal}      \\
      \begin{itemize}
            \item Ausgang \textbf{unabhängig} von zukünftigen Werten
            \item $y(t) = f(x(t-t_0)) $
      \end{itemize}
                             &
      \begin{itemize}
            \item Ausgang \textbf{abhängig} von zukünftigem Eingang
            \item $y(t) = f(x(t+t_0)) $
      \end{itemize}
      \\
      \hline
      \rowcolor{TabularTitleColor}
      \textbf{Linear}        & \textbf{Nicht-Linear} \\
      \begin{itemize}
            \item Linearkombination am Eingang -> Linearkombination am Ausgang
            \item Lineare DGL mit konstanten Koeffizenten
            \item $\Phi(\alpha x_a(t)+\beta x_b(t)) =\alpha y_a(t) + \beta y_b(t)$
            \item $\Phi(0) = 0$
      \end{itemize}
                             &
      \begin{itemize}
            \item Ausgang ist nicht immer Proportional zum Eingang
            \item Keine Superposition
            \item Neue Frequenzanteile werden erzeugt
      \end{itemize}

      \\
      \rowcolor{TabularTitleColor}
      \textbf{Zeitinvariant} & \textbf{Zeitvariant}  \\
      \begin{itemize}
            \item Zeitliche Verschiebung am Eingang -> gleiche Zeitliche Verschiebung am Ausgang
            \item $\Phi(x(t-t_v)) = y(t-t_v)$
      \end{itemize}
                             &
      \begin{itemize}
            \item Zeitinvarianz ist nicht erfüllt
            \item y(t) enthält t unabhägig von x(t)
      \end{itemize}
      \\
      \hline
\end{tabular}
\section{Wichtige Seiten im Skript:}
\titlewithref{Klassifizierung von Modellen}{100} \\
\titlewithref{Linearisierung von Systemen}{93,95}
\subsection{\titlewithref{Schaltende und stetigähnliche Regler}{61, 71}}
\titlewithref{Geschaltete Steuerung}{62} \\
\titlewithref{Zweipunktregler}{63}
\titlewithref{Zweipunktregler mit interner Rückkopplung}{71} \\
\titlewithref{Schrittregler}{73}
\subsection{\titlewithref{Modellierung von Regelstrecken}{42}}
\titlewithref{Füllprozess}{42} \\
\titlewithref{elektrischer Heizprozess}{47} \\
\titlewithref{Gleichstromantrieb}{52}\\
\titlewithref{Auto (Vorwärtsbewegung)}{57}\\
\titlewithref{Identifikation weiterer Prozesse}{59}

\newpage
\input{include/Integraltransformationen/Integraltransformationen.tex}
%needs Packages:
% - \usepackage[export]{adjustbox} for  "valign=t"

\section{Wichtige Funktionen}
\begin{tabular}{p{5cm} p{12.5cm}}
  \includegraphics[width = 5cm, valign=t]{include/Wichtige Funktionen/img/Sprungfunktion.png} &
  \textbf{Sprungfunktion (Heaviside)}
  \footnotesize
  Normierter Einschaltvorgang
  $$H(t) = \begin{cases}
               0 \textrm{ für }  t<0,                                                                      \\
               [\frac{1}{2} \textrm{ für }  t = 0,] \textrm{ \tiny(nicht immer vorhanden; dann 1 für t=0)} \\
               1 \textrm{ für }  t >0.
             \end{cases}   $$
  \\
  \includegraphics[width = 5cm, valign=t]{include/Wichtige Funktionen/img/Impulsfunktion.png} &
  \textbf{Diracimpuls} \tiny (auch Impuls-/Deltafunktion, Delta-Distribution)
  \footnotesize \newline
  Unendlicher kurzer normierter Impuls mit unendlicher Amplitude
  $$\int\limits _{-\infty} ^{+\infty} f(t) \cdot \delta (t-t_0) dt = f(t_0) \;
    \int\limits _{-\infty} ^{+\infty} f(t) \cdot \delta (t) dt = f(0) \;
    \int\limits _{-\infty} ^{+\infty} \delta (t) dt = 1$$
  TODO: \includegraphics[width=5cm]{include/Wichtige Funktionen/img/Eigenschaften _delta.png}   \\
  \includegraphics[width = 5cm, valign=t]{include/Wichtige Funktionen/img/Signumfunktion.png} &
  \textbf{Signumfunktion} (Vorzeichenfunktion)
  \footnotesize
  $$sgn(t) = \begin{cases}
                 -1 \textrm{ für }  t<0,  \\
                 0 \textrm{ für }  t = 0, \\
                 1 \textrm{ für }  t >0.
               \end{cases}   $$                                                           \\
  \includegraphics[width=5cm, valign=t]{include/Wichtige Funktionen/img/Rampenfunktion.png}   &
  \textbf{Rampenfunktion}
  \footnotesize
  $$r(t) = \begin{cases}
               0 \textrm{ für } t \leq 0, \\
               t \textrm{ für } t > 0.
             \end{cases}$$                                                           \\
  \includegraphics[width=5cm, valign=t]{include/Wichtige Funktionen/img/Rechteckimpuls.png}   &
  \textbf{Rechteckimpuls}
  \footnotesize
  $$p_a(t) = u(t+a)-u(t-a)= \begin{cases}
                                1 \textrm{ für } |t| < a,           \\
                                \frac{1}{2} \textrm{ für } |t| = a, \\
                                0 \textrm{ für } |t| > a.
                              \end{cases} $$                                 \\
  \includegraphics[width=5cm, valign=t]{include/Wichtige Funktionen/img/Dreieckimpuls.png}    &
  \textbf{Dreieckimpuls}
  \footnotesize
  $$\Lambda(t) = \begin{cases}
                     1 - \frac{|t|}{a} \textrm{ für } |t| < a \\
                     0 \textrm{ für } |t| \geq a
                   \end{cases}$$                                       \\
  \includegraphics[width=5cm, valign=t]{include/Wichtige Funktionen/img/SincFunktion.png}     &
  \textbf{Sinc-Funktion}
  \footnotesize
  $$sinc(t) = \frac{sin(t)}{t} \forall t$$                                                                                     \\
\end{tabular}

\input{include/LTI-Systeme_kurz/LTI-Systeme_kurz.tex}
\input{include/Integrieren und Differenzieren/Integrieren und Differenzieren.tex}
\section{Tabellen}

\subsubsection*{Blockschaltbilder MatLab}
\begin{tabular}{|c|c|c|c|c|}
      \hline
      \textbf{Summierer}                                            &
      \textbf{Differenzbilder}                                      &
      \textbf{Konstante}                                            &
      \textbf{Proportionalglied}                                    &
      \textbf{Integrierer}                                            \\
      \includegraphics[]{img/matlab/sum_block_icon.png}             &
      \includegraphics[]{img/matlab/difference_block_icon.png}      &
      \includegraphics[]{img/matlab/constant_block_icon.png}        &
      \includegraphics[]{img/matlab/gain_block_icon.png}            &
      \includegraphics[]{img/matlab/integrator_block_icon.png}        \\
      \hline
      \textbf{Totzeitglied}                                         &
      \textbf{Differenzierer}                                       &
      \textbf{}                                                     &
      \textbf{}                                                     &
      \textbf{}                                                       \\
      \includegraphics[]{img/matlab/transport_delay_block_icon.png} &
      \includegraphics[]{img/matlab/derivative_block_icon.png}      &
                                                                    &
                                                                    &
      \\
      \hline
\end{tabular}

\includepdf[pages=-]{AnhangPDF/fourierLaplaceTabelle_v02.pdf}

\end{document}
