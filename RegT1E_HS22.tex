% !TeX program = xelatex
% !TeX encoding = utf8
% !TeX root = RegT1E_HS22.tex

%% TODO: publish to CTAN
\documentclass[margin=normal]{tex/hsrzf}

%%%%%%%%%%%%%%%%%%%%%%%%%%%%%%%%%%%%%%%%%%%%%%%%%%%
% Packages

%% TODO: publish to CTAN
\usepackage{tex/hsrstud}

%% Language configuration
\usepackage{polyglossia}
\setdefaultlanguage[variant=swiss]{german}

%% License configuration
\usepackage[
    type={CC},
    modifier={by-nc-sa},
    version={4.0},
    lang={german},
]{doclicense}

%%%%%%%%%%%%%%%%%%%%%%%%%%%%%%%%%%%%%%%%%%%%%%%%%%%
% Metadata

\course{Elektrotechnik}
\module{RegT1E}
\semester{Herbstsemester 2022}

\authoremail{joel.leirer@ost.ch}
\author{\textsl{Joël Leirer} -- \texttt{\theauthoremail}}

% did someone help you with this work?
\contributors{
  % I created this template, does that count?
  Naoki Pross
  % do not forget to add yourself!
}

\title{\texttt{\themodule} Zusammenfassung}
\date{\thesemester}

%%%%%%%%%%%%%%%%%%%%%%%%%%%%%%%%%%%%%%%%%%%%%%%%%%%
% Document

\begin{document}

% use roman numberals for introductiory pages
\pagenumbering{roman}

\maketitle


% show the names of the people who contributed to this document.
% \section*{Contributors}
% \thecontributors

\section*{Lizenz}
\doclicenseThis

\tableofcontents

% actual content
\clearpage
\setcounter{page}{1}
\pagenumbering{arabic}

\section{Begriffe}
\begin{itemize}
  \item Prozess
        \begin{itemize}
          \item Die Gesamtheit zusammenwirkender Vorgänge, welche durch
                die Materie, Energie und Information
                umgefort, transportiert und gespeichert wird.
        \end{itemize}

  \item System
        \begin{itemize}
          \item Ist gegenüber der Umwelt abgegrenzt, hat Eingänge,
                Ausgänge (und Zustand).
        \end{itemize}

  \item Modell
        \begin{itemize}
          \item Beschreibung von Systemen, wird genutzt für
                Erklärung, Prognose, Gestaltung und Optimierung.
          \item Es gibt kein "richtiges Modell",
                ein Modell beschreibt ein System nur so genau wie nötig.
        \end{itemize}
  \item Modellieren
        \begin{itemize}
          \item Teilsysteme erstellen und diese
                in weitere Teilsysteme aufzutrennen.
                So erhält man einfache Grundsysteme (Grundglieder),
                welche sich einfach Mathematisch beschreiben lassen.
                Zusätzlich kehern einige Teilsysteme in ihrer
                Struktur oft wieder und können wiederverwendet werden.

          \item Top-Down: System in Teilsysteme teilen
          \item Bottom-Up: System aus Grundglieder aufbauen
        \end{itemize}
  \item Grundglieder
        \begin{itemize}
          \item Kleinste Teilsysteme. Sie können weiter augeteilt werden in:
          \item Statische Grundglieder(Systeme ohne Gedächtnis):
                Der Ausgang ist nur vom aktuellen Eingang abhängig.
                $y(t)=f(u(t))$
          \item Dynamische Glieder (System mit Gedächtnis):
                Der Ausgang ist vom aktuellen sowie von vergangenen Eingängen Abhängig.
                Vergangene Eingänge werden im System als Zustand x(t) gespeichert.
                $y(t)=f(u(t),x(t))$
        \end{itemize}
\end{itemize}


\end{document}
