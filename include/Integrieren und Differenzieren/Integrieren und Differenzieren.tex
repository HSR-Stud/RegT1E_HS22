%%%%%%%%%%%%%%%%%%%%%%%%%%%%%%%%%%%%%%%%%%%%%%
%Includes + Defines
%%%%%%%%%%%%%%%%%%%%%%%%%%%%%%%%%%%%%%%%%%%%%%

%\usepackage{color, colortbl}
%\usepackage{trfsigns}
%\usepackage{graphicx}
%\definecolor{TabularBackgroundColor}{rgb}{0.83,0.96,0.96}


%%%%%%%%%%%%%%%%%%%%%%%%%%%%%%%%%%%%%%%%%%%%%%
%Content
%%%%%%%%%%%%%%%%%%%%%%%%%%%%%%%%%%%%%%%%%%%%%%


\section{Integrieren und Differenzieren}
\subsection{Integrationsregeln}
\begin{tabular}{ll}
    Linearit\"at & $\int{f(\alpha x+\beta )dx=\frac{1}{\alpha}\cdot F(\alpha x+\beta)+C}$ \\
    Partielle Integration & $\int\limits_a^b{u'(x)\cdot v(x)dx}=\biggl[
    u(x)\cdot v(x) \biggr]_a^b-\int\limits_a^b{u(x)\cdot v'(x)dx}$ 
    \tiny($v(x)$ = einfacheste Funktion wählen!) \normalsize\\
    
    Substitution (Rationalisierung) & $t=\tan\frac{x}{2}, \qquad
    dx=\frac{2dt}{1+t^2} \qquad \sin  x=\frac{2t}{1+t^2} \qquad \cos x=\frac{1-t^2}{1+t^2}
    \quad\int{R(\sin(x)\cos(x))dx}$\\

    Allgemeine Substitution &
    $\int\limits_{a}^{b}{f(x)dx}=\int\limits_{g(a)}^{g(b)}{f(g(t))\cdot
    g'(t)dt}\qquad x=g(t)\qquad g'(t)=\frac{dt}{dx}\qquad dx=\frac{1}{g'(t)}\cdot dt$\\
    
    Logarithmische Integration & $\int{\frac{f'(x)}{f(x)}dx}=\ln|f(x)|+C 
    \qquad{(f(x)\neq 1)}$\\

    Spezielle Form des Integranden & $\int{f'(x)\cdot
    (f(x))^{\alpha} dx}= f(x)^{\alpha +1}\cdot \frac{1}{\alpha+1}+C
    \qquad{(\alpha \neq -1)}$\\

    Differentiation & $\int \limits ^{b} _{a} {f'(t)dt}=f(b)-f(a)$\qquad
    $\frac{d}{dx} \int \limits ^{x} _{1} {f(t)dt}=f(x)$
  \end{tabular}

\subsection{Ableitungsregeln}
\begin{tabular}{ll}
Linerität & $(\lambda f + \mu g)'(x) = \lambda f'(x) + \mu g'(x) \; \forall \lambda, \mu \in \mathbb{R} $ \\
Produktregel & $(f\cdot g)' = f'g + fg'$\\
Quotientenregel & $(\frac{f}{g})' = \frac{f'g - fg'}{g^2}$\\
Kettenregel & $ (f(g))' = f'(g) \cdot g' $\\
Potez & $((x-a)^n)'= n\cdot(x-a)^{n-1}$\\
Trigo & $sin'''' = cos''' = -sin'' = -cos' = sin$\\
\end{tabular}


\subsection{Partialbruchzerlegung}
Nenner Faktorisieren, bei mehreren gleichen Termen steigt exponent des Nenners
Zähler = Polynom mit einem Grad kleiner als Nenner
Zähler Gleichsetzen, Mit Gleichungssystem nach A,B,C.. auflösen.
Ansätze:
$$\frac{\dots}{x(x-3)^2} = \frac{A}{x} + \frac{B}{x-3} + \frac{C}{(x-3)^2}$$
$$\frac{\dots}{(x-2)^3} = \frac{A}{x-2} + \frac{B}{(x-2)^2} + \frac{C}{(x-2)^3}$$
$$\frac{\dots}{x^4 + x^2} = \frac{A}{x} + \frac{B}{x^2} + \frac{Cx + D}{x^2 + 1}$$